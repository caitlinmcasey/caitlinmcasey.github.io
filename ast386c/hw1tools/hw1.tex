\documentclass[11pt]{article}
% \usepackage{phase1}
\usepackage{graphics,graphicx}
\usepackage[margin=1in]{geometry}
\usepackage{fancyhdr}
\usepackage{wrapfig}
\usepackage{epsfig}
\usepackage{amssymb}
\usepackage{wrapfig}
\usepackage{setspace}
\usepackage[labelfont=bf]{caption}
%\usepackage{enumitem}
\usepackage{hyperref}
\def\Vhrulefill{\leavevmode\leaders\hrule height 0.7ex depth \dimexpr0.4pt-0.7ex\hfill\kern0pt}
\usepackage{amstext}
\usepackage{array}
\newcolumntype{C}{>{$}c<{$}}

\newcommand{\etal}{et~al.}

\begin{document}

\begin{center}
\textbf{\large Homework \#1: Conversions and Stellar Population Synthesis}
\vspace{2mm}

{\sc AST386c: Prof.\ Caitlin Casey}
\end{center}
\vspace{4mm}

\noindent This assignment is due by 3\,pm, Tuesday, September
25$^{th}$ (i.e. no missing colloquium for this!).  Send me a pdf (in a
format lastname\_hw1.pdf) in email to cmcasey`at'utexas.edu.  It
should be formatted using \LaTeX\footnote{If you don't know
  \LaTeX\ yet, don't sweat it!  Come let me know and I'd be happy to
  show you the basics.  Also talk to your peers! There is a lot of
  collective knowledge in this building that be really helpful to tap
  into. }, showing your work, and embedding plots as needed.  You can
generate the plots in any code you prefer, but please include a full
description of your methods so I could reproduce your plot exactly
following your instructions.

\vspace{4mm}

\begin{enumerate}
\item Download all of the goodies available at the following link:\\
  {\tt www.as.utexas.edu/$\sim$cmcasey/ast386c/hw1tools/}
\item Converting between flux, flux density, luminosity and magnitudes
  are critical skills to professional astronomers, and often prove
  confusing.  Here are a set of exercises to make you more adept at
  these conversions.  In the various downloaded tools for this problem
  set you will find the following files:
  \begin{center}
\vspace{-0.5cm}
  \begin{tabular}{|l|l|}
%    \multicolumn{2}{c}{www.as.utexas.edu/$\sim$cmcasey/ast386/hw1tools/}\\
    \hline
    Template Stellar A0V Spectrum & spectrum\_A0V.txt \\
    \hline
    Subaru $g$-band filter profile & subaru\_g.txt \\
    \hline
    Subaru $r$-band filter profile & subaru\_r.txt \\
    \hline
    Subaru $i$-band filter profile & subaru\_i.txt \\
    \hline
    Subaru $z$-band filter profile & subaru\_z.txt \\
    \hline
    Subaru $Y$-band filter profile & subaru\_y.txt \\
    \hline
  \end{tabular}\\
  \end{center}
  The first column of every file is the wavelength (in \AA).  The
  second column of the template stellar spectrum is given in $S_{\nu}$
  units (erg\,s$^{-1}$\,cm$^{-2}$\,Hz$^{-1}$), and the second column
  of the filter profile is representative of the total system response
  T$_\lambda$, or the product of the filter response, detector quantum
  efficiency, instrument and telescope throughput and atomsphereic
  transmission.  In other words, T$_\lambda$ is the fraction of light
  that gets through at a given wavelength.
  \begin{enumerate}
    \item Plot the spectrum of the A0V star, along with a
      flat-spectrum source of flux density 3631\,Jy, overplotting and
      labeling each of these filters.  Make sure the plot is legible,
      and add labels/legends sufficient for me to fully understand
      what is plotted.  You will need to rescale the filter curves to
      be clearly visible on the plot.
    \item the AB magnitude system is defined such that a flat-spectrum
      source with flux density of 3631\,Jy has a measured flux density
      of (...drumroll...) 3631\,Jy in all filters, regardless of filter
      bandwidth/shape.  In the $AB$ magnitude system, such a source
      would have a magnitude of 0 in all bands.  If the template
      spectrum for the A0V star were to represent Vega, what is the
      magnitude of Vega in the AB system across these five filters?
      (This is the offset between Vega magnitudes and AB magnitudes
      for these filters.)
      \item What do you notice about the values of these offsets and
        the shape of the spectrum of Vega?
      \item Plot the A0V spectrum in units of $\nu\!L_{\nu}$, assuming
        it is Vega and Vega is 7.68\,pc from us (note that
        $1\,pc=3.086\times10^{16}\,m$).  Overplot
        $\lambda\!L_{\lambda}$ on the same plot for comparison.  From
        this plot, what can you surmise about how intrinsically bright
        Vega is compared to the Sun? (this should give you some
        feeling of why people sometimes plot in units of
        $\nu\!L_{\nu}$ or $\lambda\!L_\lambda$.)
  \end{enumerate}


\item This problem will introduce the concept of stellar population
  synthesis (SPS) models by building up a basic understanding of
  stellar populations.  SPS models are critical to how we understand
  the stellar emission of galaxies as integrated light sources
  (because in the vast majority of cases we cannot see individual
  stars in other galaxies!), and so it is very important to understand
  how they are built.  This problem deals with bolometric quantities,
  and the next problem will introduce some of the mechanics of
  building SPS models with real templates.
  \begin{enumerate}
    \item The Salpeter IMF (Salpeter 1955) is parameterized:
      \begin{equation}
        \xi(\log\!m) = \frac{d(N/V)}{d\log\!m} = \frac{dn}{d\log\!m}\propto\!m^{-x}
      \end{equation}
      where $x=1.35$.  Using this distribution of stellar masses, plot
      the cumulative stellar mass fraction from high masses to low, in
      other words $f(>\!m)$ vs. $m$.  You can stop at the brown
      dwarf/hydrogen burning limit, $\sim$80\,M$_{\rm jup}$ and at
      100\,M$_\odot$ at the high-mass end. What is the average mass of
      a star drawn at random from this Salpeter distribution (i.e. the
      expectation value)?
    \item The relationship between a star's luminosity and mass can be
      parameterized roughly as:
      \vspace{1mm}
      \begin{tabular}{CCCC}
        \frac{L}{L_\odot}\approx0.23\Big(\frac{M}{M_\odot}\Big)^{2.3} & 
        \frac{L}{L_\odot}\approx\Big(\frac{M}{M_\odot}\Big)^{4} & 
        \frac{L}{L_\odot}\approx1.5\Big(\frac{M}{M_\odot}\Big)^{3.5} & 
        \frac{L}{L_\odot}\approx3200\Big(\frac{M}{M_\odot}\Big) \\
        (M<0.43M_\odot) &
        (0.43M_\odot<M<2M_\odot) &
        (2M_\odot<M<20M_\odot) &
        (M>20M_\odot) \\
      \end{tabular}
      \vspace{1mm} Remember that the luminosity is proportional to the
      rate of fuel consumption and the mass of the star is
      proportional to the total fuel supply, and that the sun's
      lifetime is $\sim$10\,Gyr.  What is the highest mass main
      sequence star you expect to live past 100\,Myr? 500\,Myr?
      1\,Gyr?
    \item If you assume that a stellar population is aged 500\,Myr,
      what would be the average mass of a star drawn at random?  How
      does this differ from your answer to part (a), and how would it
      differ for a stellar population that is aged 1\,Gyr? (You can
      ignore evolved stars for the purposes of this problem even
      though we know they're... important.)
    \item What is the fractional contribution for stars of a given
      mass $m$ to the total light emitted by a given stellar
      population?  Plot this as a cumulative distribution, $f_{\rm
        L}(>m)$ as a function of $m$.  Hint: it would be wise to
      convert $\xi(\log\!m)$ to $dn/dl$ for this step.  What can you
      say about what types of stars dominate the light of any given
      stellar population?
    \item Generate the same plot in part (d) but adjust it to
      represent a 500\,Myr-old and 1\,Gyr-old stellar population.
      Mark the three curves (including the 0-age curve) clearly.
\end{enumerate}
  \item This problem builds on the previous problem, but now you are
    asked to build up a stellar population model spectrally.  To do
    this you will need to check out / unpack the following files from
    the directory linked above: kurucz93.tar.gz and
    EEM\_dwarf\_UBVIJHK\_colors\_Teff.txt.  The first is a directory
    of stellar atmosphere model spectra from Kurucz (1993) for a range
    of metallicities, effective temperatures and surface gravities.
    The txt file should be used to map effective temperature and
    luminosity back to mass\footnote{The table quotes T$_{\rm eff}$
      for stars between $0.1<M/M_\odot<19.6$ with a rough gridding, so
      you'll have to interpolate between these points to come up with
      good T$_{\rm eff}$ estimates for all mass points on the scale
      you used in problem 1.  At high masses you should interpolate
      towards a 100\,M$_\odot$ star having a 45,000\,K temperature.}.
    You can read the readme files in the Kurucz directories for more
    information, in addition to the annotations of the txt file.
  \begin{enumerate}
  \item Using the same mass range as in problem 1, plot $T_{\rm
    eff}$ against stellar mass and stellar luminosity $L$ vs stellar
    mass (by interpolating the values given in the reference txt
    file).
  \item Now that you know the effective temperature and luminosity
    across our entire mass range, you can make a composite spectrum
    for the stellar population as a whole.  For each value of your
    stellar mass grid, you should read in the appropriate Kurucz
    model, choosing the closest one in effective temperature. For
    simplicity, just assume solar metallicity for all stars (only the
    files in the kp00 directory). For the gravity you can adopt:
    $T_{\rm eff}\ge41000\,K$ (column g50), $36000\ge T_{\rm
      eff}<41000\,K$ (column g45), $9000\ge T_{\rm eff}<36000\,K$
    (column g40), and $T_{\rm eff}<9000\,K$ (column g45).  You'll want
    to add all of the spectra of the stars together, proportional to
    how many stars of each type are in the stellar population.  Plot
    the resulting stellar population spectrum in $\nu L_\nu$ units
    against wavelength.  Be sure to make sensible choices for your
    axes and think about whether or not it would be best to use a
    linear or a log scale to present your results.
  \item Now go back and split up this spectrum into the contribution
    from stars in different mass ranges.  You choose 3-4 mass ranges
    that you think convey the most interesting results.  State what
    mass ranges you assume clearly and label them on your plot.
  \item Using your results from problem 1 now generate a spectrum of
    this stellar population after it has aged 500\,Myr, and dilineate
    the contributions from each of the mass ranges chosen in part
    (c).  Then, do the same for a 1\,Gyr age.
  \item What differences and similarities do you notice across these
    stellar population models?  What types of stars dominate the
    spectrum over what wavelength range?
  \end{enumerate}
  Great! You made it to the end of Homework \#1.  Congrats!  Now keep
  your stellar population models handy... we will use them in the
  beginning of Homework \#2!
\end{enumerate}


\end{document}
